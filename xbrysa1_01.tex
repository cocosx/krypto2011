\documentclass[a4paper,11pt]{article}
\usepackage[utf8]{inputenc}
\usepackage[IL2]{fontenc}
\usepackage{listings}
\usepackage{amsmath}
\usepackage{amssymb}
\usepackage{fullpage}
%\topmargin -2cm%
\textheight 24cm
\setlength\parindent{0pt}
\thispagestyle{empty}
\begin{document}
\begin{flushleft}
Číslo cvičení: 1 \\ 
Jméno: Marek Bryša \\
UČO: 323771\\
Login: xbrysa1\\
\end{flushleft}
\begin{enumerate}
\item $h(C)=4$, $s(C)=3$, $t(C)=1$.\\C is equivalent to code $D=\{0000000,1111000,1100111,001111\}.$ (The last bit of all codewords is negated.)
\item Let the first codeword WLOG be $00000$. The other codewords must differ in at least 4 bits, therefore can be from $A=\{01111,10111,11011,11101,11110,11111\}$.\\$max\{h(x,y)| x,y\in A\}=2\implies$ the third codeword can only have a distance of 2 from the first two $\implies$ a binary $(5,3,4)$ code does not exist.
\item
\item The code is defined by the following table.
  \begin{center}
  \begin{tabular}{|c|c|c|c|c|}
    \hline
    A&B&C&D&E\\\hline
    1&2&3&44&43\\\hline
  \end{tabular}
  \end{center}
  The second symbol for the D and E characters could be chosen. It is probably better to choose the symbols that are statistically least used in the rest of the code.
\item For any codeword of length $n-1$, there is exactly 1 symbol that can be put on the last position so that the resulting codeword of length $n$ is in $C$ $\implies M=q^{(n-1)}$. Because of that, if one symbol is changed, another must also be changed. Such change can be always made $\implies  d=2$.
\item
  \begin{enumerate}
  \item 8 distinct messages $\implies n\geq 3$. $n=3$ does not provide room for any error correction. For $n=4$, one additional bit is not enought to correct a shift of two bits. The following code provides the requested correction for $n=5$.
  \begin{center}
  \begin{tabular}{|c|c|c|c|}
    \hline
    Msg & unchanged & $p_l$ & $p_r$ \\\hline
    A & 00000 &&\\
    B & 11111 &&\\
    C & 10000 & 00001 & 01000 \\
    D & 11000 & 10001 & 01100 \\
    E & 11100 & 11001 & 01110 \\
    F & 11110 & 11101 & 01111 \\
    G & 10100 & 01001 & 01010 \\
    H & 10101 & 01011 & 11010 \\\hline    
  \end{tabular}
  \end{center}
  The unchanged message is sent. The receiver decodes it using any of unchanged, $p_l$, $p_r$.
  \item In a 4-bit code, there are 16 total codewords, 4 are symmetric and are always transmitted correctly. For the reamining $16-4=12$ we need two codewords for each message $\implies$ the total maximum is $4+(16-4)/2=10$.
  \end{enumerate}
\item Only (b) is a possible Huffman code. (a) is not a prefix code. (c),(d) unnecessarily add a trailing 0 to their last codeword.
\item $x=9$, Cryptography: An Introduction.
\end{enumerate}
\end{document}
