\documentclass[a4wide,12pt]{extarticle}
\usepackage[utf8]{inputenc}
\usepackage[IL2]{fontenc}
\usepackage{listings}
\usepackage{amsmath}
\usepackage{amssymb}
%\usepackage{fullpage}
\usepackage[top=15mm, bottom=15mm, left=15mm, right=15mm]{geometry}
\setlength\parindent{0pt}
\thispagestyle{empty}
\begin{document}
\begin{flushleft}
Sada: 1 \\ 
Jméno: Marek Bryša \\
UČO: 323771\\
\end{flushleft}
\begin{enumerate}
  \item
		Redukcí z 3-SAT. Stroj $M$ pro jazyk $H$ má jako vstup kódování 3-SAT forumule $\varphi$. Vyzkouší všech $2^n$ vstupů na formuli $\varphi$, a pokud je pro některý splněna, akceptuje, jinak cyklí. To samo o sobě nemá polynomiální složitost, ale redukční funkce pouze $f$ generuje kód stroje $M$ a to zvládne lineárně vzhledem k velikosti forumule $\varphi$. QED
	\item
	\item
		Problém černého cyklu (BC) je NP-úplný.
		\begin{itemize}
			\item BC $\in$ NP: Analogicky jako HAM se algoritmus nedeterministicky rozhodne pro počateční vrchol a navazující vrcholy a kontroluje splnění podmínky.
			\item
				BC je NP-těžký: Redukcí z HAM\\
				$f$ přidá ke grafu $G(v,e)$ $|v|$ dalších samostatně ležících vrcholů, popř. k některému vrcholu připojí "had" délky $|v|$. $f$ je lineární vzhledem k velikosti $G$.\\
				$G\in$HAM $\Rightarrow f(G)\in$BC: Pokud byl v $G$ ham. cyklus, bude v $f(G)$ zřejmě právě jeden černý cyklus, a to ten, který byl v $G$ hamiltonovský.\\
				$G\in$HAM $\Leftarrow f(G)\in$BC: Ten jediný černý cyklus v $f(G)$ bude zároveň hamiltonovský v $G$.
				
		\end{itemize}
		 QED
	\item
	\item
		$f$ přidá ke $G$ kliku velikosti 6 a spojí 3 její vrcholy se všemi vrcholy $G$. Pro každý vrchol $G$ tedy zbudou právě 3 možné barvy. $f$ je lineární vzhledem k velikosti $G$. Implikace "$\Rightarrow$" i "$\Leftarrow$" zřejmě platí. QED
	\item
		
\end{enumerate}
\end{document}
