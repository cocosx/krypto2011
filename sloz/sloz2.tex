\documentclass[a4wide,10pt]{extarticle}
\usepackage[utf8]{inputenc}
\usepackage[IL2]{fontenc}
\usepackage{listings}
\usepackage{amsmath}
\usepackage{amssymb}
%\usepackage{fullpage}
\usepackage[top=15mm, bottom=15mm, left=15mm, right=15mm]{geometry}
\setlength\parindent{0pt}
\thispagestyle{empty}
\begin{document}
\begin{flushleft}
Sada 2, Marek Bryša, 323771
\end{flushleft}
\begin{enumerate}
  \item
		Redukcí z 3-SAT. Stroj $M$ pro jazyk $H$ má jako vstup kódování 3-SAT forumule $\varphi$. Vyzkouší všech $2^n$ vstupů na formuli $\varphi$, a pokud je pro některý splněna, akceptuje, jinak cyklí. To samo o sobě nemá polynomiální složitost, ale redukční funkce pouze $f$ generuje kód stroje $M$ a to zvládne lineárně vzhledem k velikosti forumule $\varphi$. QED
	\item
		Existuje polynomiální algoritmus. Za univerzálně kvantifikované proměnné dosadí 0, za existenčně kvantifikové 1 a vyhodnotí formuli. Platnost tohoto postupu lze ověřit při kvantifikaci $\exists e_1,e_2 \forall u_1,u_2$ na formulích $e_1 \wedge e_1, e_1 \vee e_1, e_1 \wedge e_2, e_1 \vee e_2, u_1 \wedge u_1, \dots, u_1 \wedge e_1, \dots$, přičemž složitější formule plynou ze strukturální indukce.
	\item
		Problém černého cyklu (BC) je NP-úplný.
		\begin{itemize}
			\item BC $\in$ NP: Analogicky jako HAM se algoritmus nedeterministicky rozhodne pro počateční vrchol a navazující vrcholy a kontroluje splnění podmínky.
			\item
				BC je NP-těžký: Redukcí z HAM\\
				$f$ přidá ke grafu $G(v,e)$ $|v|$ dalších samostatně ležících vrcholů, popř. k některému vrcholu připojí "had" délky $|v|$. $f$ je lineární vzhledem k velikosti $G$.\\
				$G\in$HAM $\Rightarrow f(G)\in$BC: Pokud byl v $G$ ham. cyklus, bude v $f(G)$ zřejmě právě jeden černý cyklus, a to ten, který byl v $G$ hamiltonovský.\\
				$G\in$HAM $\Leftarrow f(G)\in$BC: Ten jediný černý cyklus v $f(G)$ bude zároveň hamiltonovský v $G$.
				
		\end{itemize}
	\item
		Kameny $\in$ NP: Algoritmus nedeterministicky volí následující tah a kontroluje výhru. Pro každý tah musí zvolit z $|v|$ vrcholů a max. $|e|$ hran, tedy je celkově NP.\\
		Kameny je NP-těžký: Redukcí z HAM. $f$ zvolí "náhodně" vrchol $v_0$ z $G(v,e)$ s nejmenším počtem navazujících hran a provede:
		\begin{enumerate}
			\item Pokud $v_0$ má max. jednu hranu, $G\not\in $ HAM, tedy na všechny vrcholy umístí 0 kamenů.
			\item Pokud $v_0$ má 2 a více hran, přidá 2 vrcholy $v_1,v_2$ spojené hranou, spojí $v_1$ se sousedy $v_0$, umístí na $v_0$ 2 kameny, na ostatní včetně $v_1,v_2$ jeden kámen.
		\end{enumerate}
		V případě (b) je zajištěno, že hra začne ve $v_0$, protože má jako jediný 2 kameny a bude pokračovat na další vrcholy, přičemž každý navštíví max. jednou. Pokud $G\in$ HAM, existuje taková posloupnost tahů, že se  hra vrátí do sousedů $v_0$, pokračuje do $v_1$ a skončí úspěšně ve $v_2$. Naopak taková úspěšná hra je svědkem existence HC v $G$. Přidané "přemostění" (soused $v_0\leftrightarrow v_1 \leftrightarrow $ jiný soused $v_0$) na tom nic nemění díky přidání "ocasu" $v_2$. $f$ je vzhledem k velikosti $G$ zřejmě lineární.
	\item
		$f$ přidá ke $G$ kliku velikosti 6 a spojí 3 její vrcholy se všemi vrcholy $G$. Pro každý vrchol $G$ tedy zbudou právě 3 možné barvy. $f$ je lineární vzhledem k velikosti $G$. Implikace "$\Rightarrow$" i "$\Leftarrow$" zřejmě platí.
	\item
		Redukcí z HAM. SHAM $\in$ NP zřejmě -- oproti HAM stačí navíc nedeterministicky volit hranu, která jako jediná dostane ohodnocení 1, zbylé 0. $f$ provede jedno z:
		\begin{enumerate}
			\item Pokud v $G$ existuje osamocený vrchol nebo vrchol s jen jednou hranou, nedělá nic, protože takový graf nemá HC, tedy ani SHC.
			\item Pokud v $G$ existuje vrchol se dvěma hranami, jedné z nich dá ohodnocení 1, všem zbylým 0, neboť pokud v $G$ HC existuje, musí jednou z těchto hran procházet.
			\item Pokud v $G$ mají všechny vrcholy minimálně 3 a více hran, vybere "náhodně" jeden vrchol $v$ s nejmenším počtem hran a rozdělí ho na 2 vrcholy $v_1, v_2$ spojené jednou hranou a oba dále spojí se všemí vrcholy, se kterými byl spojen $v$. Hraně spojující $v_1, v_2$ přiřadí ohodnocení 1, všem ostatním 0. Pokud v $G$ existoval HC, musel procházet vrcholem $v$. Nyní SHC v $f(G)$ musí procházet hranou spojující $v_1, v_2$.
		\end{enumerate}
		$f$ je lineární vzhledem k velikosti $G$. Korektnost obou implikací redukce pro body (a), (b) je zřejmá z předchozí argumentace a toho, že neměníme strukturu grafu. U (c) byla zdůvodněna implikace "$\Rightarrow$". Zbývá vsvětlit "$\Leftarrow$", tedy že rozdělením vrcholu $v$ na dva nevznikne HC, který v původním $G$ nebyl. Může sice vzniknout, ale nebude obsahovat hranu mezi $v_1, v_2$ s ohodnocením 1, tudíž nebude SHC. Tuto hranu nebude obsahovat, protože jinak by měl i původní graf HC.
\end{enumerate}
\end{document}
