\documentclass{beamer}


\usepackage[czech]{babel}
\usepackage{amsthm, amsfonts, enumitem}
\newtheorem*{thm}{Teor\'em}
\newtheorem*{mydef}{Defin\'icia}
\newtheorem*{exampl}{Pr\'iklad}

\usetheme{Boadilla} 

\begin{document}
\title{Solutions}
\author{Michal Abaffy}
\institute{P\v{r}F MU}
\frame{\titlepage}




\begin{frame}{Hry v strategickej forme}

\v{C}as\v{t} 1: Ka\v{z}d\'y s\'am za seba

\begin{mydef}
Nech $X$ je mno\v{z}ina strat\'egii $i$-teho hr\'a\v{c}a a $Z$ je mno\v{z}ina strat\'egii ostatn\'ych hr\'a\v{c}ov. Nech $u_{i}: X \times Z \to \mathbb{R}$ je \'u\v{z}itkov\'a funkcia $i$-teho hr\'a\v{c}a. Strat\'egia $x \in X$ {\bf dominuje} strat\'egii $y \in X$ ak pre v\v{s}etky mo\v{z}n\'e strat\'egie $z \in Z$ plat\'i $u_{i}(x,z) \geq u_{i}(y,z)$ a existuje strat\'egia $z' \in Z$ tak\'a, \v{z}e $u_{i}(x,z') > u_{i}(y,z')$.

\end{mydef}

\begin{mydef}
Strat\'egia $x\in X$ sa naz\'yva {\bf nedominovan\'a}, ak neexistuje strat\'egia $x' \in X$, ktor\'a by jej dominovala.
\end{mydef}

V hr\'ach, kde ka\v{z}d\'y hr\'a s\'am za seba nem\'a zmysel hra\v{t} dominovan\'e strat\'egie.

\end{frame}


\begin{frame}{Hry v strategickej forme}

\begin{mydef}
Nech $X$ je mno\v{z}ina strat\'egii hr\'a\v{c}ov, nech $u_{i}$ je \'u\v{z}itkov\'a funkcia $i$-teho hr\'a\v{c}a. Situ\'acia $x \in X$ {\bf dominuje pod\v{l}a Pareta} situ\'acii $x' \in X$, ak \[\forall i: u_{i}(x) \geq u_{i}(x') \textrm{ a } \exists i: u_{i}(x) > u_{i}(x')\]

\end{mydef}

\begin{mydef}
Situ\'acia $x\in X$ sa naz\'yva {\bf optim\'alna pod\v{l}a Pareta}, ak neexistuje situ\'acia, ktor\'a by jej dominovala pod\v{l}a Pareta. 
\end{mydef}

Niektor\'e situ\'acie optim\'alne pod\v{l}a Pareta mo\v{z}u prinies\v{t} niektor\'emu z hr\'a\v{c}ov menej, ako si s\'am dok\'a\v{z}e zaru\v{c}i\v{t}. Vi\v{d} pr\'iklad neskor. 

Mo\v{z}nost zavies\v{t} \emph{imputation set}, kde s\'u tie situ\'acie, ktor\'e s\'u optim\'alne pod\v{l}a Pareta a ka\v{z}d\'y hr\'a\v{c} dostane aspo\v{n} to\v{l}ko, ko\v{l}ko si vie s\'am garantova\v{t}.
\end{frame}


\begin{frame}{Hry v strategickej forme}

\begin{mydef}
Nech $X_{i}$ je mno\v{z}ina strat\'egii $i$-teho hr\'a\v{c}a a $Z_{i}$ je mno\v{z}ina strat\'egii ostatn\'ych hr\'a\v{c}ov. Nech $u_{i}: X_{i} \times Z_{i} \to \mathbb{R}$ je \'u\v{z}itkov\'a funkcia $i$-teho hr\'a\v{c}a. Situ\'acia $x=(y,z), y\in X_{i}, z\in Z_{i}$ sa naz\'yva {\bf rovnov\'a\v{z}na pod\v{l}a Nasha}, ak plat\'i
\[\forall i, \forall y' \in X_{i}: u_{i}(x) \geq u_{i}(y',z)\]
\end{mydef}

Pretlmo\v{c}en\'e: Situ\'acia je rovnov\'a\v{z}na pod\v{l}a Nasha, ak si \v{z}iaden z hr\'a\v{c}ov s\'am nepomo\v{z}e k lep\v{s}ej v\'yhre. V [5], Nash dok\'azal nasleduj\'uci teor\'em.

\begin{thm}
Existuj\'u hry, kde pri \v{c}ist\'ych strat\'egi\'ach neexistuje situ\'acia RPN, av\v{s}ak pri zmie\v{s}an\'ych strat\'egi\'ach m\'a ka\v{z}d\'a hra situ\'aciu RPN.
\end{thm}

\end{frame}


\begin{frame}{Hry v strategickej forme}

\begin{exampl} Uva\v{z}ujme nasleduj\'uce dve hry dvoch hr\'a\v{c}ov v strategickej forme:

\

\begin{tabular}{ l l l l}
Strat\'egie            & 1 & 2 & 3 \\
\hline
 1  & (1,6) & (0,2) & (1,3) \\
 2  & (2,3) & (2,2) & (2,3) \\
\end{tabular}

\

\

\begin{tabular}{l l l}
Strat\'egie            & 1 & 2 \\
\hline
 1  & (6,6) & (2,0) \\
 2  & (0,2) & (5,5) \\
\end{tabular}

\

Pre obe hry. N\'ajdite nedominovan\'e strat\'egie pre oboch hr\'a\v{c}ov, n\'ajdite v\v{s}etky situ\'acie optim\'alne pod\v{l}a Pareta a rovnov\'a\v{z}ne pod\v{l}a Nasha a imputation sety.

\end{exampl}

\end{frame}



\begin{frame}{Koali\v{c}n\'e hry}

\v{C}as\v{t} 2: Koali\v{c}n\'e hry

Predpoklady:

- M\'ame mno\v{z}inu hr\'a\v{c}ov $N$. Majme \v{l}ubovoln\'u koal\'iciu $S \subseteq N$. Uva\v{z}ujme, \v{z}e hr\'a\v{c}i z $S$ navz\'ajom spolupracuj\'u. Mo\v{z}u zvoli\v{t} spolo\v{c}n\'u strat\'egiu, ktor\'a im garantuje ist\'u sumu $v(S)$, o ktor\'u sa n\'asledne mo\v{z}u ist\'ym sposobom rozdeli\v{t}. 

- Technick\'y detail: v\'yhry musia by\v{t} v spolo\v{c}nej mene (alebo aspo\v{n} prevediteln\'e do spolo\v{c}nej meny), a ka\v{z}d\'y hr\'a\v{c} si ka\v{z}d\'u v\'yhru cen\'i rovnako (\'u\v{z}itkov\'e funkcie roznych hr\'a\v{c}ov su rovnak\'e).

- Zjavne: $S \cap T = \emptyset \implies v(S \cup T) \geq v(S) + v(T)$.

- Po\v{z}adujeme: $v(\emptyset)=0$.

\

Aj in\'y typ koali\v{c}n\'ych hier: hr\'a\v{c}i mo\v{z}u spolupracova\v{t} v \v{l}ubovolnej koal\'icii, ale v\'yhry nie s\'u prenosn\'e. S tak\'ymto typom hier sa pracuje zlo\v{z}itej\v{s}ie, my sa nimi nebudeme zaobera\v{t}.

\end{frame}




\begin{frame}{Jadro (The core)}

Chcen\'e rie\v{s}enie probl\'emu (nemus\'i existova\v{t}): pre ka\v{z}d\'u koal\'iciu plat\'i, \v{z}e s\'u\v{c}et v\'yhier jej hr\'a\v{c}ov je aspo\v{n} to\v{l}ko, ko\v{l}ko garantuje \'u\v{c}as\v{t} v koal\'icii. Form\'alne:

Rie\v{s}enie je {\bf payoff vector} $(x_{1}, ..., x_{n})$, kde $x_{i}$ predstavuje v\'yhru $i$-teho hr\'a\v{c}a a plat\'i $\sum_{i=1}^{n} x_{i} = v(N)$.

\begin{mydef}
Jadro s\'u tak\'e payoff vectory, kde plat\'i:
\[\forall S \subseteq N: \sum_{i \in S}x_{i} \geq v(S)\]
\end{mydef}


\end{frame}


\begin{frame}{Jadro (The core)}

\begin{exampl} Nie v\v{z}dy mus\'i tak\'eto rie\v{s}enie existova\v{t}. 

$v(\emptyset)=0,$

$ v(\{1\})=v(\{2\})=v(\{3\})=0,$

$v(\{1,2\})=v(\{1,3\})=v(\{2,3\})=1, $

$v(\{1,2,3\})=1$.

\end{exampl}


\end{frame}




\begin{frame}{Jadro (The core)}

V jednoduch\'ych pr\'ipadoch rie\v{s}ime tak, \v{z}e zvol\'ime koal\'iciu v\v{s}etk\'ych hr\'a\v{c}ov a nakreslen\'im.

\begin{exampl}
$v(\emptyset)= v(\{1\})=v(\{2\})=v(\{3\})=0,$

$v(\{1,2\})= 1, v(\{1,3\})=2, v(\{2,3\})=3$

$v(\{1,2,3\})= 4$

\end{exampl}

Rie\v{s}enie: nakresl\'ime trojuholn\'ik $(4,0,0), (0,4,0), (0,0,4)$ a vkres\v{l}ujeme do\v{n} nerovnosti.

V pr\'ipade, \v{z}e jadro neexistuje, mo\v{z}eme chcie\v{t} od rie\v{s}enia aspo\v{n}, aby to bol takzvan\'y {\bf rozumn\'y payoff vector}, teda by malo sp\'l\v{n}a\v{t}: 
\[ \forall i: x_{i} \leq \max_{S}\{v(S)-v(S-\{i\})\}\]
\end{frame}




\begin{frame}{Jadro (The core)}

V\'yhody: 

- Satisfy all subgroup rationality, i.e., no subgroup is offered less than it could obtain by itself.

\

Nev\'yhody: 

- Jadro nemus\'i existova\v{t}.

- Mo\v{z}e n\'ajs\v{t} mno\v{z}inu rie\v{s}en\'i a nie je jasn\'e, ktor\'y hr\'a\v{c} ko\v{l}ko dostane.

\

Daj\'u sa \v{c}iasto\v{c}ne odstr\'ani\v{t} (no vznikn\'u nov\'e nedostatky) zaveden\'im siln\'eho (slab\'eho) $\epsilon$-core, kde rie\v{s}enie $x$ mus\'i sp\'l\v{n}a\v{t}:
\[\forall S \subseteq N: \sum_{i \in S}x_{i} \geq v(S) - \epsilon  \textrm{, respekt\'ive }\] 
\[\forall S \subseteq N: \sum_{i \in S}x_{i} \geq v(S) - |S|\epsilon \]


\end{frame}



\begin{frame}{Simple games}

Simple games sp\'l\v{n}aj\'u nasleduj\'uce axi\'omy:

(1) $\forall S \subseteq N: v(S)=0$ or $v(S)=1$.

(2) $v(\emptyset)=0$

(3) $v(N)=1$

(4) No losing set contains a winning subset. (Pod\v{l}a m\v{n}a zbyto\v{c}n\'y, plynie z axi\'omov pre charakteristick\'u funkciu $v$.)

\

V\'yznamn\'a podtrieda simple games ({\bf rozhoduj\'uce SG}) sp\'l\v{n}a i nasledovn\'y axi\'om:

(5)  $\forall S \subseteq N: v(S) + v(N - S)= 1$.

\end{frame}




\begin{frame}{Shapley vector}

Jednoduch\'e hry va\v{c}\v{s}inou nemaj\'u rie\v{s}enie v podobe jadra.

Rie\v{s}enie (nielen pre RSG): {\bf Shapleyho vector $x$}
\[x_{i}(v)=\sum_{S \subseteq N}\frac{(n-s)!(s-1)!}{n!}(v(S)-v(S-\{i\}))\]

Predpoklady ved\'uce k Shapleyho vectoru: aditivita na nez\'avisl\'ych hr\'ach; symetria hr\'a\v{c}ov; linearita pri n\'asoben\'i skal\'arom (hra a vektor); $\sum_{i \in S} x_{i}(u) = u(S)$, kde $S \subseteq N$ je \v{l}ubovoln\'a koal\'icia, ktor\'a obsahuje v\v{s}etk\'ych podstatn\'ych hr\'a\v{c}ov (hr\'a\v{c} $i$ je podstatn\'y, ak existuje koal\'icia $S$, t\v{z} $v(S \cup \{i\})> v(S) + v(\{i\})$).

\

Interpret\'acia: Each individual is assumed to enter every possible coalition in every way randomly, and he is then assigned the expected value of the incremental gain he brings to all.

\end{frame}




\begin{frame}{Shapley vector}

\begin{exampl}
Vo\v{l}by do parlamentu. Zvolen\'ych 5 str\'an a nech $A,B,C,D,E$ zna\v{c}ia po\v{c}et poslancov jednotliv\'ych str\'an. $A=55, B=30, C=25, D=22, E=18$. V\'i\v{t}azn\'e koal\'icie s\'u tie, ktor\'e maj\'u viac ako 75 poslancov. Spo\v{c}\'itajte SV.
\end{exampl}

\

Rie\v{s}enie na \v{d}al\v{s}ej strane. 

\end{frame}




\begin{frame}{Shapley vector}

\emph{Rie\v{s}enie:} V\'i\v{t}azne koal\'icie (s \v{c}iarkou s\'u zna\v{c}en\'i t\'i, ktor\'ych pr\'ichod do koal\'icie zmen\'i nev\'i\v{t}azn\'u koal\'iciu na v\'i\v{t}zn\'u): $A'B', A'C', A'D', A'BC, A'BD, A'B'E, A'CD, A'C'E, A'D'E, B'C'D', ABCD,$ $ A'BCE, A'BDE, A'CDE, B'C'D'E, ABCDE$.

$x_{A} = 3 \frac{3!1!}{5!} + 6 \frac{2!2!}{5!} + 3 \frac{1!3!}{5!}= \frac{18+24+18}{120}=\frac{1}{2}$

$x_{B} = 1 \frac{3!1!}{5!} + 2 \frac{2!2!}{5!} + 1 \frac{1!3!}{5!}= \frac{6+8+6}{120} = \frac{1}{6}$

$x_{C} = 1 \frac{3!1!}{5!} + 2 \frac{2!2!}{5!} + 1 \frac{1!3!}{5!}= \frac{6+8+6}{120} = \frac{1}{6}$

$x_{D} = 1 \frac{3!1!}{5!} + 2 \frac{2!2!}{5!} + 1 \frac{1!3!}{5!}= \frac{6+8+6}{120} = \frac{1}{6}$

$x_{E} = 0$

Kontrola: $x_{A}+x_{B}+x_{C}+x_{D}+x_{E}=1=v(N)$.



\end{frame}




\begin{frame}{Stable set}

Predpoklad: plne kooperat\'ivna hra, zvol\'i sa ve\v{l}k\'a koal\'icia a jej zisk sa distribuuje pomocou payoff vectoru medzi hr\'a\v{c}ov.
 
\begin{mydef}
{\bf Imputation} je tak\'y payoff vector $x$, kde \v{z}iaden hr\'a\v{c} nedostane menej ako si dok\'aze s\'am zaru\v{c}i\v{t}. 
\[ \forall i:x_{i} \geq v(\{i\})\]
\end{mydef}


\begin{mydef}
Majme hru $(N,v)$ a nech $x,y$ s\'u dve imputations $(N,v)$. {\bf $x$ dominuje $y$}, ak $\exists S \neq \emptyset$, t\v{z} $\forall i \in S :x_{i} > y_{i}$ a $\sum_{i \in S} x_{i} \leq v(S)$ (Pre\v{c}o nie rad\v{s}ej  $\sum_{i \in S} y_{i} < v(S)$? ).	
\end{mydef}

 In other words, players in $S$ prefer the payoffs from $x$ to those from $y$, and they can threaten to leave the grand coalition if $y$ is used because the payoff they obtain on their own is at least as large as the allocation they receive under $x$. 

\end{frame}



\begin{frame}{Stable set}

\begin{mydef}
A {\bf stable set} is a set of imputations that satisfies two properties:

Internal stability: No payoff vector in the stable set is dominated by another vector in the set.

External stability: All payoff vectors outside the set are dominated by at least one vector in the set.
\end{mydef}

Von Neumann povodne myslel, \v{z}e ka\v{z}d\'a plne kooperat\'ivna koali\v{c}n\'a hra m\'a stable set rie\v{s}enie. Nie je tomu tak.

\end{frame}


\begin{frame}{Jadro (The kernel)}

\begin{mydef}
Majme hru $(N,v)$ a payoff vector $x$.
\[s_{ij}^{v}(x):=\max_{S \subseteq N-\{j\}; i \in S}\{v(S) - \sum_{k \in S}x_{k}\}\]
\end{mydef}

\v{C}o $s_{ij}^{v}(x)$ predstavuje? 

Predpokladajme, ze $j$ chce vyst\'upi\v{t} z ve\v{l}kej koal\'icie, kde sa prerozde\v{l}uje pod payoff vectorom $x$. Predpokladajme, \v{z}e ostatn\'i hr\'a\v{c}i s\'u spokojn\'i so svojou \v{c}as\v{t}ou $x_{k}$. Tak\v{z}e ak $j$ sa rozhodne nespolupracova\v{t}, hr\'a\v{c} $i$ mo\v{z}e z\'iska\v{t} maxim\'alne $s_{ij}^{v}(x)$, ak ostatn\'i hr\'a\v{c}i bud\'u s\'uhlasi\v{t} s vytvoren\'im koal\'icie S, kde ich zisk zo zisku koal\'icie bude rovn\'y ich zisku v povodnom payoff vectore. 

Intuit\'ivne, ak $s_{ij}^{v}(x) > s_{ji}^{v}(x)$, $i$ m\'a va\v{c}\v{s}iu vyjedn\'avaciu silu ako $j$ pri payoff vectore $x$.

\end{frame}



\begin{frame}{Jadro (The kernel)}

Od dobr\'eho rie\v{s}enia o\v{c}ak\'avame, \v{z}e ka\v{z}d\'a dvojica hr\'a\v{c}ov bude ma\v{t} vo\v{c}i sebe rovnak\'u vyjedn\'avaciu silu. Plat\'i ale, \v{z}e hr\'a\v{c} $i$ je odoln\'y vo\v{c}i hrozb\'am in\'ych hr\'a\v{c}ov, ak $x_{i}-v(\{i\})=0$, preto\v{z}e mo\v{z}e zdrhn\'u\v{t} a spravi\'t si vlastn\'u koal\'iciu.

\begin{mydef}
Jadro hry obsahuje tak\'e imputations, kde \v{z}iadny hr\'a\v{c} nem\'a nad \v{z}iadnym hr\'a\v{c}om vyjedn\'avaciu silu, 
\[ [s_{ij}^{v}(x) - s_{ji}^{v}(x)][x_{j}-v(\{j\})] \leq 0 \textrm{ and}\]
\[ [s_{ji}^{v}(x) - s_{ij}^{v}(x)][x_{i}-v(\{i\})] \leq 0 \]

\end{mydef}

Plat\'i nasleduj\'uca veta [4]:

\begin{thm}
Ka\v{z}d\'a koali\v{c}n\'a hra m\'a nepr\'azdny kernel. 
\end{thm}

\end{frame}


\begin{frame}{References}

[1] K. J. Arrow, M. D. Intriligator, Handbook of Mathematical Economics, Vol. 1, ISBN: 0444861262

[2] Wikipedia: Imputation (game theory), Cooperative game, Core (game theory), Transferable utility

[3] L. Pol\'ak: Teorie her, u\v{c}ebn\'y text predmetu Teorie her, Masarykova univerzita

[4] M. Maschler and B. Peleg, Pacific J. Math. Volume 18, Number 2 (1966), 289-328.

[5] J. Nash, "Equilibrium points in n-person games", Proceedings of the National Academy of Sciences 36(1), (1950), 48-49.

\end{frame}



\end{document}