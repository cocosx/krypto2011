\documentclass[a4paper,10pt]{extarticle}
\usepackage[utf8]{inputenc}
\usepackage[IL2]{fontenc}
\usepackage{listings}
\usepackage{amsmath}
\usepackage{amssymb}
\usepackage{fullpage}
%\topmargin -2cm%
\textheight 24cm
\setlength\parindent{0pt}
\thispagestyle{empty}
\begin{document}
\begin{flushleft}
Číslo cvičení: 3 \\ 
Jméno: Marek Bryša \\
UČO: 323771\\
Login: xbrysa1\\
\end{flushleft}
\begin{enumerate}
  \item
    \begin{enumerate}
      \item
        $x^6-1 = (1-x+x^2)(1+x+x^3+x^4)\implies h(x)=x^2-x+1$
      \item
        $(x^2-x+1)(x^5+x^4+x^3)=x(-1+x^6)+(x+x^3+x^5), (x^2-x+1)(x^5+x^4+x^3+x)=x(-1+x^6)+(2x-x^2+2x^3+x^5) \implies$ the words do not belong to $C$.
    \end{enumerate}
  \item
    \begin{enumerate}
      \item
        Not linear $\implies$ not cyclic, not equivalent to a cyclic code.
      \item
        Not a cyclic code, not equivalent to a cyclic code.
      \item
        Is a cyclic code.
      \item
        Not a cyclic code, not equivalent to a cyclic code.
    \end{enumerate}
  \item
  \item
    MDS $\iff M=q^{n-d+1}$. Ham$(r,2)$ has a $(r\times (2^r-1))=((n-k)\times n)$ parity check matrix $\implies n=2^r-1, n-k=r, d=3$. $M=q^k=2^{n-r}$.
    \begin{align}
      2^{2^r -1-3+1}&=2^{2^r -1-r}\nonumber\\
      2^r -3&=2^r-1-r\nonumber\\
      r&=2\nonumber
    \end{align}
  \item
    $(x^7-1)/(x^3+x+1)=x^4+x^2+x+1=h(x)$.\\
    \[
      H=
      \begin{pmatrix}
        1&1&1&0&1&0&0\\
        0&1&1&1&0&1&0\\
        0&0&1&1&1&0&1
      \end{pmatrix}
      , \bar{h}(x)=x^4+x^3+x^2+1
    \]
  \item
    $x^6-1=(x-1)(x+1)(x^2-x+1)(x^2+x+1)$. There are $2^4=16$ such codes.
    Their generator polynomials are: $1,(x-1),(x+1),(x^2-x+1),(x^2+x+1),
    (x^2-1),(x^3-2x^2+2x-1),(x^3-1),(x^3+1),(x^3+2x^2+2x+1),(x^4+x^2+1),
    (x^4+x^3-x+1),(x^4-x^3+x-1),
    (x^5-x^4+x^3-x^2+x-1),(x^5+x^4+x^3+x^2+x+1),(x^6-1)=0$\\
    \begin{center}
      \begin{tabular}{|c|c|}
        \hline
        Polynomial & Matrix \\\hline
        $1$ & $I_6$\\\hline
        $x+1$ &
          $\begin{bmatrix}
            1&1&0&0&0&0\\
            0&1&1&0&0&0\\
            0&0&1&1&0&0\\
            0&0&0&1&1&0\\
            0&0&0&0&1&1\\
          \end{bmatrix}$
          \\
        \hline
        $x^2+1$ &
          $\begin{bmatrix}
            1&0&1&0&0&0\\
            0&1&0&1&0&0\\
            0&0&1&0&1&0\\
            0&0&0&1&0&1\\
          \end{bmatrix}$
          \\
        \hline
        $x^3+1$ &
          $\begin{bmatrix}
            1&0&0&1&0&0\\
            0&1&0&0&1&0\\
            0&0&1&0&0&1\\
          \end{bmatrix}$
          \\
        \hline
        $x^4+x^2+1$ &
          $\begin{bmatrix}
            1&0&1&0&1&0\\
            0&1&0&1&0&1\\
          \end{bmatrix}$
          \\
        \hline
        $(x^5+x^4+x^3+x^2+x+1)$ &
          $\begin{bmatrix}
            1&1&1&1&1&1\\
          \end{bmatrix}$
          \\
        \hline
      \end{tabular}
    \end{center}
  \item
    \begin{enumerate}
      \item
        Let $C_1$ be a repetition code of length $n$, $C_2$ a no-parity code of length $n$ $\implies$ $\neg C_1=C_1, C_1\cap C_2=C_1=C_3$.
      \item
        $C_3$ does not exist. There must be at least one non-zero codeword in $C_1$. If $C_3$ were to be cyclic, that non-zero bit would have to be shifted to all positions including the last one.
      \item
        Let $C_1$ and $C_2$ be the same as in (a) $\implies C_1 \cup C_2=C_2=C_3$.
    \end{enumerate}
\end{enumerate}
\end{document}
