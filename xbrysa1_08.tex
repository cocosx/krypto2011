\documentclass[a4paper,10pt]{extarticle}
\usepackage[utf8]{inputenc}
\usepackage[IL2]{fontenc}
\usepackage{listings}
\usepackage{amsmath}
\usepackage{amssymb}
\usepackage{fullpage}
%\topmargin -2cm%
\textheight 24cm
\setlength\parindent{0pt}
\thispagestyle{empty}
\begin{document}
\begin{flushleft}
Číslo cvičení: 8 \\ 
Jméno: Marek Bryša \\
UČO: 323771\\
Login: xbrysa1\\
\end{flushleft}
\begin{enumerate}
  \item
    \[
    n^{40}+1=(n^8)^5+1=x^5+1=(x+1)\cdot(x^4-x^3+x^2-x+1)
    \]
  \item
    $4a^3+27b^2 \mod p \neq 0 \iff$ elliptic curve can be used to form a group over $F_p$.\\
    $4\cdot 1000+27\cdot25 \mod 17=0\implies$ the e.c. does not form the group.
  \item
    \begin{enumerate}
      \item
        $a_1=3^2+1 \mod 4577=10$, $b_1=(3^2+1 \mod 4577)^2 +1 \mod 4577=101$, $gcd(10-101,4577)=1$\\
        $a_2=10^2+1 \mod 4577=101$, $b_1=(101^2+1 \mod 4577)^2 +1 \mod 4577=4402$, $gcd(101-4402,4577)=23$\\
        $4577=23\cdot199$
      \item
        $2P=(80,65)$ and we compute $gcd(283,143)=1$.\\
        $3P=(131,102)$ and we compute $gcd(64,143)=1$.\\
        $4P=(14,28)$ and we compute $gcd(13,143)=13$.
    \end{enumerate}
  \item
    The elliptic curve is isomorphic to $Z_5$. $\infty$ has the role of 0.
    \begin{center}
      \begin{tabular}{|c|ccccc|}
        \hline
        %%%%%%%%%%%%%%%%%%%%%%%%%%%%%%%%%%%%%%%%%%%%%%%%%%%%%%%%%%%%%%%%%%%%%%
%%                                                                  %%
%%  This is a LaTeX2e table fragment exported from Gnumeric.        %%
%%                                                                  %%
%%%%%%%%%%%%%%%%%%%%%%%%%%%%%%%%%%%%%%%%%%%%%%%%%%%%%%%%%%%%%%%%%%%%%%
+	&$\infty$	&(0,1)	&(0,6)	&(4,2)	&(4,5)\\\hline
$\infty$	&$\infty$	&(0,1)	&(0,6)	&(4,2)	&(4,5)\\
(0,1)	&(0,1)	&(4,5)	&$\infty$	&(0,6)	&(4,2)\\
(0,6)	&(0,6)	&$\infty$	&(4,2)	&(4,5)	&(0,1)\\
(4,2)	&(4,2)	&(0,6)	&(4,5)	&(0,1)	&$\infty$\\
(4,5)	&(4,5)	&(4,2)	&(0,1)	&$\infty$	&(0,6)\\\hline

      \end{tabular}
    \end{center}
  \item
   $x^{11}-1=0 \mod 2011\iff x^{11}=1 \mod 2011$. From Euler's theorem $a^{\varphi(n)}=1 \mod n$ if $a$ is coprime to $n$
   and here all are, since $n$ is prime. $\varphi(2011)=2010=2\cdot3\cdot5\cdot67$. In case of a prime exponent, the solution to the original equations other than $x=1$ only exist for prime factors of 2010. 11 is not one of them so the solution does not exist.
\end{enumerate}
\end{document}
