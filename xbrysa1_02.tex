\documentclass[a4paper,11pt]{article}
\usepackage[utf8]{inputenc}
\usepackage[IL2]{fontenc}
\usepackage{listings}
\usepackage{amsmath}
\usepackage{amssymb}
\usepackage{fullpage}
%\topmargin -2cm%
\textheight 24cm
\setlength\parindent{0pt}
\thispagestyle{empty}
\begin{document}
\begin{flushleft}
Číslo cvičení: 2 \\ 
Jméno: Marek Bryša \\
UČO: 323771\\
Login: xbrysa1\\
\end{flushleft}
\begin{enumerate}
  \item
    \begin{enumerate}
      \item
        \[
          P=\begin{pmatrix}
            1 & 1 & 1 & 0 \\
            1 & 2 & 0 & 1 \\
          \end{pmatrix}
        \]
      \item
        $C=\{0000,1022,0121,1110,2102,1201,2011,0212,2220\}$

    \end{enumerate}
  \item
    \begin{enumerate}
      \item
        Basis of such code can be in the form $(AA)$, $(0A)$ or $(A0)$, the last two being equivalent. The respective codes are therefore: $C_1=(00,11,22,33)$, $C_2=(00,01,02,03)$. Hence possible values for $d$ are 1, 2.
      \item
        Basis of such code is in the form $G=\begin{pmatrix}
            0 & 1 \\
            1 & 0 \\
          \end{pmatrix}$. The only possible $q$ to generate exactly 4 codewords is 2. $C=\{00,01,10,11\}$.

    \end{enumerate}
  \item
    \begin{enumerate}
      \item
        Because all operations are applied separately on columns "concateration" of two codes results in $n=n_1+n_2$, $k=k_1=k_2$, $d=d_1+d_2$.
      \item
        $n=n_1+n_2$. All rows in $G'$ are linearly independent, so $k=k_1+k_2$.
        Codewords generated in the form $x_1\dots x_{n_1}0\dots 0$ will exist in the resulting code. 
        Similarly $0\dots 0x_1\dots x_{n_2}$. Therefore $d=min\{d_1, d_2\}$.
    \end{enumerate}
    \item
    \item
      $R(1,m)$ is obtained by "adding" repetition codes to $R(1,1)$ like this: $R(1,1)+R(0,1)=R(1,2)$, $R(1,2)+R(0,2)=R(1,3), \dots$
      Nuber of cosets generaly equals to $q^{n-k}$. $q=2, n=2^m$.
    \item
      \begin{enumerate}
        \item
          
      \end{enumerate}
\end{enumerate}
\end{document}
