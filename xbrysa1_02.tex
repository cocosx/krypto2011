\documentclass[a4paper,9pt]{extarticle}
\usepackage[utf8]{inputenc}
\usepackage[IL2]{fontenc}
\usepackage{listings}
\usepackage{amsmath}
\usepackage{amssymb}
\usepackage{fullpage}
%\topmargin -2cm%
\textheight 24cm
\setlength\parindent{0pt}
\thispagestyle{empty}
\begin{document}
\begin{flushleft}
Číslo cvičení: 2 \\ 
Jméno: Marek Bryša \\
UČO: 323771\\
Login: xbrysa1\\
\end{flushleft}
\begin{enumerate}
  \item
    \begin{enumerate}
      \item
        \[
          P=\begin{pmatrix}
            1 & 1 & 1 & 0 \\
            1 & 2 & 0 & 1 \\
          \end{pmatrix}
        \]
      \item
        $C=\{0000,1022,0121,1110,2102,1201,2011,0212,2220\}$
        \begin{center}
          \begin{tabular}{|c|c|c|c|c|c|c|c|c||c|}
            \hline
            0000	&1022	&0121	&1110	&2102	&1201	&2011	&0212	&2220	&00\\
            \hline
            1000	&2022	&1121	&2110	&0102	&2201	&0011	&1212	&0220	&11\\
            0100	&1122	&0221	&1210	&2202	&1001	&2111	&0012	&2020	&12\\
            0010	&1002	&0101	&1120	&2112	&1211	&2021	&0222	&2200	&10\\
            0001	&1020	&0122	&1111	&2100	&1202	&2012	&0210	&2221	&01\\
            2000	&0022	&2121	&0110	&1102	&0201	&1011	&2212	&1220	&22\\
            0200	&1222	&0021	&1010	&2002	&1101	&2211	&0112	&2120	&21\\
            0020	&1012	&0111	&1100	&2122	&1221	&2001	&0202	&2210	&20\\
            0002	&1021	&0120	&1112	&2101	&1200	&2010	&0211	&2222	&02\\
            \hline
          \end{tabular}
        \end{center}
      \item
        0201$\rightarrow$1201, 1111$\rightarrow$1110.

    \end{enumerate}
  \item
    \begin{enumerate}
      \item
        Basis of such code can be in the form $(AA)$, $(0A)$ or $(A0)$, the last two being equivalent. The respective codes are therefore: $C_1=(00,11,22,33)$, $C_2=(00,01,02,03)$. Hence possible values for $d$ are 1, 2.
      \item
        Basis of such code is in the form $G=\begin{pmatrix}
            0 & 1 \\
            1 & 0 \\
          \end{pmatrix}$. The only possible $q$ to generate exactly 4 codewords is 2. $C=\{00,01,10,11\}$.

    \end{enumerate}
  \item
    \begin{enumerate}
      \item
        Because all operations are applied separately on columns "concateration" of two codes results in $n=n_1+n_2$, $k=k_1=k_2$, $d=d_1+d_2$.
      \item
        $n=n_1+n_2$.\\
        All rows in $G'$ are linearly independent, so $k=k_1+k_2$.
        Codewords generated in the form $x_1\dots x_{n_1}0\dots 0$ will exist in the resulting code. 
        Similarly $0\dots 0x_1\dots x_{n_2}$. Therefore $d=min\{d_1, d_2\}$.
    \end{enumerate}
  \item
    \begin{enumerate}
      \item
        \[
          G\sim
          \begin{pmatrix}
            0&1&0&0&1\\
            0&2&0&1&0\\
            1&1&1&0&0\\
          \end{pmatrix}
          \sim
          \begin{pmatrix}
            1&0&0&0&1\\
            0&1&0&0&2\\
            0&0&1&1&1\\
          \end{pmatrix}
        \]
      \item
        $C$ has $q^k=3^3=27$ codewords.
      \item
        $w(C)=2=h(C)<3\Rightarrow C$ cannot correct any errors.
        
    \end{enumerate}

  \item
    $R(1,m)$ is obtained by "adding" repetition codes to $R(1,1)$ like this: $R(1,1)+R(0,1)=R(1,2)$, $R(1,2)+R(0,2)=R(1,3), \dots$
    Nuber of cosets generaly equals to $q^{n-k}$. $q=2, n=2^m, k=1+\binom{m}{1}=1+m\Rightarrow$ number of cosets $=2^{2^m-m-1}$.
  \item
    \begin{enumerate}
      \item
        Both conditions of the linear code definition are made using bit-by-bit operations. Removing one bit therefore cannot break any of them.
      \item
        $n^i=n-1$.\\
        If the removed coordinate has a "twin" bit in the basis (i.e. all bits in a repetition code are "twins"), $k^i=k$.
        If the basis is like $G$ from 2. (a),$k^i=k-1$. No other options are possible, because removing one coordinate cannot decrease dimension by 2 or more.\\
        For similar reasons $d^i=d$ or $d^i=d-1$.
        
    \end{enumerate}
  \item Such codes can be generated by this general basis matrix of size $\frac{n}{2}$ by $n$:
    \[ 
      G=
      \begin{pmatrix}
        1 & 1 & 0 & 0 & \cdots & 0 \\
        0 & 1 & 1 & 0 & \cdots & 0 \\
        \vdots &   &   & \ddots \\
        0 & \cdots & 0 & 0 & 1 & 1 \\
      \end{pmatrix}
    \]
\end{enumerate}
\end{document}
