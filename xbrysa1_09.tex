\documentclass[a4paper,10pt]{extarticle}
\usepackage[utf8]{inputenc}
\usepackage[IL2]{fontenc}
\usepackage{listings}
\usepackage{amsmath}
\usepackage{amssymb}
\usepackage{fullpage}
%\topmargin -2cm%
\textheight 24cm
\setlength\parindent{0pt}
\thispagestyle{empty}
\begin{document}
\begin{flushleft}
Číslo cvičení: 9 \\ 
Jméno: Marek Bryša \\
UČO: 323771\\
Login: xbrysa1\\
\end{flushleft}
\begin{enumerate}
  \item
    \begin{enumerate}
      \item
        $4586^{107}=1 \mod 7919$, $q$ is prime.
      \item
        $v=4586^{-55}=1175 \mod 7919$
      \item
        $\gamma=4586^{29}=48 \mod 7919$
      \item
        $y=29+55\cdot61=3384 \mod 7919$
      \item
        $4586^{3384}\cdot1175^{61}=48=\gamma \mod 7919$
    \end{enumerate}
  \item
    The father can use Shamir's $(5,3)$-threshold scheme, where the eldest son receives 2 pieces of the secret and the others get 1 each.
  \item
    \begin{enumerate}
      \item
        Not a MAC. Considering a chosen message attack we would authentize $(a||b,e_k(a||b))$ and $(c||d,e_k(c||d))$. Then we could forge $a||d$ and $c||b$. The same applies if a man-in-the-middle intercepts two different messages.
      \item
        Not a MAC. Reason is the same as in (a).
      \item
        Not a MAC. Intercepting a single message is sufficient to authentize any permutation of its $m_i$ submessages.
    \end{enumerate}
  \item
    $l_0=\frac{x-3}{1-3}\cdot\frac{x-7}{1-7}=x^2/12-(5 x)/6+7/4$\\
    $l_1=\frac{x-1}{3-1}\cdot\frac{x-7}{3-7}=-x^2/8+x-7/8$\\
    $l_2=\frac{x-1}{7-1}\cdot\frac{x-3}{7-3}=x^2/24-x/6+1/8$\\
    %28(x^2/12-(5 x)/6+7/4)+31(-x^2/8+x-7/8)+17(x^2/24-x/6+1/8)
    $f(x)=28l_0+31l_1+17l_2=-(5 x^2)/6+(29 x)/6+24$\\
    $S=f(0)=24$
  \item
    \begin{enumerate}
      \item
        Bob accepts iff $y^e=RX_A^f \mod n$.
      \item
        $Y=y^e=(rx_A^f)^e=RX_A^f \mod n$.
      \item
        In step (i) Eve chooses $R=X_a^{f(e-1)}$. In step (iii) Eve sends $y=X_A^f$. Then $Y=X_A^{fe}$ and $RX_A^f=X_A^{f(e-1)}X_A^f=X_A^{fe}\implies$ Bob accepts.
        
    \end{enumerate}
  \item
    For each group of $t-1$ scientists there must be at least one unique lock that they cannot open. And if a new scientist join the group, he must be able to open it. Hence there must be at least $\binom{n}{t-1}=\binom{11}{5}=462=L$ locks.\\
    Each group of $t$ scientists must be able to open the locks so at least $n-t+1$ must posses the keys. Each must therefore have $\frac{n-t+1}{n}\cdot L=252$ keys.
\end{enumerate}
\end{document}
