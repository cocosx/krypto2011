\documentclass[a4paper,10pt]{extarticle}
\usepackage[utf8]{inputenc}
\usepackage[IL2]{fontenc}
\usepackage{listings}
\usepackage{amsmath}
\usepackage{amssymb}
\usepackage{fullpage}
%\topmargin -2cm%
\textheight 24cm
\setlength\parindent{0pt}
\thispagestyle{empty}
\begin{document}
\begin{flushleft}
Číslo cvičení: 7 \\ 
Jméno: Marek Bryša \\
UČO: 323771\\
Login: xbrysa1\\
\end{flushleft}
\begin{enumerate}
  \item
  Order of any element of the group must be a factor of $\varphi(151)=150=2\cdot3\cdot5^2$ because 151 is a prime.
  Denote $A$ set of all integers $a$ such that $a$ is order of some element of $(Z^*_{151},\cdot)$. We now know that $A$ is
  a subset of the set of factors of 150. The group is isomorphic to $(Z_{150},+)$. In the latter group we can easily see that
  $1, 2, 3, 5, 6, 10, 15, 25, 30, 50, 75, 150$ are the orders of $0, 75, 50, 30, 25, 15, 10, 6, 5, 3, 2, 1$.\\
  The resulting set is therefore $A=\{1,2,3,5,6,10,15,25,30,50,75,150\}$.
  \item
    $w=-aij^{-1} \mod (p-1)$. Eve can sign only some messages --- those which she gets by choosing valid $i,j$.
  \item
    $S_1=\frac{1}{2}(\frac{23}{18}+18) \mod 2011=1006\cdot(23\cdot1229+18) \mod 2011=1071$\\
    $S_2=\frac{171}{2}(\frac{23}{18}-18) \mod 2011=171\cdot1006\cdot(23\cdot1229 -18) \mod 2011=1084$\\
    Verification: $1071^2-1974\cdot1084^2 \mod 2011=23$\\
    Decryption:$\frac{23}{1071+171^{-1}\cdot1084} \mod 2011=23\cdot1050 \mod 2011=18$
  \item
    $0=k^{-1}(55+x\cdot72) \mod 73\implies 0=55-x \mod 73 \implies x=55$\\
    Signing:\\
    $y=588^{55} \mod 877 = 546$\\
    Let $k=23$.\\
    $a=(588^{23} \mod 877) \mod 73)=52$\\
    $b=54\cdot(50+55\cdot52)\mod 73=44$\\
    Verification:\\
    $z=5$, $u_1=50\cdot5 \mod 73 =31$, $u_2=52\cdot5=41$\\
    $(588^{31}\cdot546^{41} \mod 877) \mod 73=52=a$
  \item
    If $H$ allows calculation of $w\neq w'$, $H(w)=H(w')$, Eve can send $w$ to Alice to be signed by her. Since the hashes
    are equal and only they are signed, Bob cannot differentiate between signatures of $w$ and $w'$, therefore Eve can send $w'$
    to Bob with a valid signature.\\
    SHA-1 still is considered secure in this notion, since discovery of collisions is computationally unfeasable. The best
    attacks known today produce collisions in around $2^{60}$ calculations.
  \item
    \begin{enumerate}
      \item
        $M=4$, Let $x_1=123$, $x_2=456$. Let's use the SHA1 hash function.\\
        $y_1=H^4(123)=1098614404817320799890476653834049837021396717148$\\
        $y_2=H^4(456)=1424490108233818249016946572506319050765232109793$\\
        $s_1=H^3(123)=1035282358065450305654857715659091571160656400828$\\
        $s_2=H^0(456)=456$\\
        $H^1(1035282358065450305654857715659091571160656400828)=y_1$\\
        $H^4(456)=y_2$
      \item
        Given signed message $n$, we could sign message $n+1$ by simply calculating hash of $s_1$ one more time.
      \item
        Computation complexity of this scheme is exponential with respect to N, as opposed to {Lamport} where it is linear.
    \end{enumerate}

    
\end{enumerate}
\end{document}
