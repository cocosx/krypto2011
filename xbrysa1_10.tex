\documentclass[a4paper,10pt]{extarticle}
\usepackage[utf8]{inputenc}
\usepackage[IL2]{fontenc}
\usepackage{listings}
\usepackage{amsmath}
\usepackage{amssymb}
\usepackage{fullpage}
%\topmargin -2cm%
\textheight 24cm
\setlength\parindent{0pt}
\thispagestyle{empty}
\begin{document}
\begin{flushleft}
Číslo cvičení: 10 \\ 
Jméno: Marek Bryša \\
UČO: 323771\\
Login: xbrysa1\\
\end{flushleft}
\begin{enumerate}
  \item
    The chance of me successfuly cheating $n$ rounds is $C=(1/2)^n$ so the probability of proving is $P=1-C$. Hence I need at least $-\log_{1/2}(1-x)$ rounds to prove myself with probability $x\cdot 100\%$.
  \item
  \item
    \begin{description}
      \item[Completeness]
        If the prover is honest, he can easily pass verifiers checks for both values of $\sigma$.
      \item[Zero knowledge]
        In each round, verifier only learns $\Pi$ or a hamiltonian cycle for a permutated graph. He would need both to reconstruct the original $C$. Verifier can easily simulate the protocol because he knows his $\sigma$ in advance and can therefore either respond to himself with a random $\Pi$ or a cycle to a similar random graph.
      \item[Soundness]
        If the prover is dishonest, he can do the same thing as the verifier in simulation, however this reduces to a coin flip situation as in 1., so with enough rounds, provers cheating will be probably revealed.
    \end{description}
  \item
    \begin{itemize}
      \item
        Peggy chooses a random permutation $\Pi$ and commits the permutated solved table face down.
      \item
        Victor asks Peggy to reveal one of the rows, one of the columns, one of the main $3\times 3$ boxes or the unsolved puzzle --- all after applying $\Pi$.
      \item
        Victor accepts if the chosen set contains all of $1\dots 9$ exactly once or in the last case the unsolved puzzle is indeed a permutation of the original.
    \end{itemize}
    Victor can prove Peggy is cheating with probability at least $1/28$ per round, so again with enough rounds this becomes certainty.
\end{enumerate}
\end{document}
