\documentclass[a4paper,12pt]{extarticle}
\usepackage[utf8]{inputenc}
\usepackage[IL2]{fontenc}
\usepackage{listings}
\usepackage{amsmath}
\usepackage{amssymb}
\usepackage{fullpage}
%\topmargin -2cm%
\textheight 24cm
\setlength\parindent{0pt}
\thispagestyle{empty}
\begin{document}
\begin{flushleft}
Číslo cvičení: 6 \\ 
Jméno: Marek Bryša \\
UČO: 323771\\
Login: xbrysa1\\
\end{flushleft}
\begin{enumerate}
  \item
    $n=11\cdot13 \implies$\\
    $w^2=56 \mod 11\implies w_1=1,\, w_2=10 \mod 11$\\
    $w^2=56 \mod 13\implies w_3=2,\, w_4=11 \mod 13$\\
    $w_1=11k+1=13l+2 \implies w_1=67$\\
    $w_2=11k+10=13l+2 \implies w_1=54$\\
    $w_3=11k+1=13l+11 \implies w_1=89$\\
    $w_4=11k+10=13l+1 \implies w_1=76$\\
  \item
  \item
    The only $p> 7$ such that none of 3,5,7 are its quadratic residues is 17.\\
    $15^{(17-1)/2}=15^8 = 21^8 = 35^8 = 1 \mod p$, $105^8=-1 \mod p\implies$\\
    15,21,35 are quadratic residues, 105 is not.
  \item
    $y=q^x \mod p=137565$, $a=q^r \mod p=89804$, $b=y^r w \mod p = 7512\implies c=(89804,7512)$\\
    $w=b(a^x)^{-1} \mod p=7512\cdot 22233\mod p = 15131 $
  \item
    $\lg_5 112 \mod 131$, $q=5,y=112,p=131$, $m=12$\\
    $L_1=(1, 117, 65, 7, 33, 62, 49, 100, 41, 81, 45, 25)$\\
    $L_2=(112, 101, 125, 25, 5, 1, 105, 21, 109, 48, 62, 91)$\\
    $25\in L_1,L_2 \implies i=3, j=11\implies x = 12\cdot11+3= 135$
  \item
    There are $\frac{p-1}{2}$ quadratic residues. They result from $1^2,2^2,\dots,(p-1)^2$.\\
    Since $a^2=(-a)^2$, they form pairs $1^2=(p-1),\dots,(\frac{p-1}{2})^2=(\frac{p+1}{2})^2 \mod p$.\\
    None of them are congruent $\mod p$. Let $a^2=b^2 \mod p$, $1\leq a\leq b\leq\frac{p-1}{2}$.\\
    $p|a^2-b^2=(a+b)(a-b) \implies p|(a+b) \vee p|(a-b)$. The first one is impossible because of the constrains for $a,b$.
    The second one can only hold for $a-b=0 \implies a=b$.
  \item
    $w_1=100\cdot4^{-42}=122 \mod 503$\\
    $w_2=457\cdot299^{-42}=30 \mod 503$
  \item
    \begin{enumerate}
      \item
        \[\bar{p}(n)=\frac{n!\binom{365}{n}}{365^n},\, p=1-\bar{p}\]
        $\bar{p}(45)=0.0590241\implies p(45)=0.9409759$
      \item
        $\bar{p}(31)=0.2695$, $\bar{p}(32)=0.2466\implies$ There must be at least 32 people.
    \end{enumerate}
    
    
\end{enumerate}
\end{document}
